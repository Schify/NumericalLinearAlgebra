\documentclass{article}
\usepackage[top=2cm, bottom=2cm, left=2.2cm, right=2.5cm]{geometry}
\usepackage{amsmath,amsfonts,amssymb}
\usepackage{enumerate}% http://ctan.org/pkg/enumerate

\usepackage[]{pdfcomment}
%As recommended by matlab2tikz
\usepackage{pgfplots}
\usepackage{tikz}
\usepackage{calc}
\usepackage{pgfplotstable}
\usepackage{booktabs}
\usepackage{csvsimple}
\usepackage{etoolbox}
\usepackage{biblatex}
\bibliography{refs.bib}
%% the following commands are needed for some matlab2tikz features
\usetikzlibrary{plotmarks}
\usetikzlibrary{arrows.meta,positioning}
\usepgfplotslibrary{patchplots}
\usepackage{grffile}
\usepackage{hyperref}
\usepackage{amsmath}
\usepackage{placeins,nicefrac}
\usepackage{xstring}
\usepackage[]{pgfkeys}
\usepackage{xfrac}
%\usepackage{minted}
\usetikzlibrary{external}
\usetikzlibrary{calc,math}
\tikzexternalize[prefix=extern/]
\usepgfplotslibrary{groupplots}
\usepgfplotslibrary{colormaps}
\usetikzlibrary{decorations.markings,patterns,patterns.meta,pgfplots.fillbetween}

\usepackage[ruled,vlined,linesnumbered,algo2e]{algorithm2e}
\usepackage{wrapfig}
\usepackage{subcaption}


\usepackage{wasysym}
\usepackage{animate}


\graphicspath{{../figs/}}
\newcommand{\figurescale}{1.0}

\usepackage{titlesec}
\titleformat{\subsection}{\normalfont\large\bfseries}{Task \thesubsection}{1em}{}
\titleformat{\section}{\normalfont\Large\bfseries}{Assignment \thesection}{1em}{}
\titleformat{\subsubsection}{\normalfont\bfseries}{Question \thesubsubsection}{1em}{}

\newcommand{\stilltodo}[1]{{\color{red} UNFINISHED #1}}

\makeatletter
\newcommand{\trp}{%
	{\mathpalette\@transpose{}}%
}
\newcommand*{\@transpose}[2]{%
	% #1: math style
	% #2: unused
	\raisebox{\depth}{$\m@th#1\intercal$}%
}

\makeatother
\newcommand{\diff}{\mathrm{d}}
\author{Andr\'as Schifferer, r0915705}
\title{Numerical Linear Algebra [H03G1A]\\{\LARGE HW 1}}
\date{\today} 

\pgfplotsset{plot coordinates/math parser=false}
\newlength\figureheight
\newlength\figurewidth 


\pgfqkeys{/pgf}{use fpu reciprocal/.code={%
		\def\pgfmathreciprocal@##1{%
			\begingroup
			\pgfkeys{/pgf/fpu=true,/pgf/fpu/output format=fixed}%
			\pgfmathparse{1/##1}%
			\pgfmath@smuggleone\pgfmathresult
			\endgroup
}}}%
\tikzset{
	set arrow inside/.code={\pgfqkeys{/tikz/arrow inside}{#1}},
	set arrow inside={end/.initial=>, opt/.initial=},
	/pgf/decoration/Mark/.style={
		mark/.expanded=at position #1 with
		{
			\noexpand\arrow[\pgfkeysvalueof{/tikz/arrow inside/opt}]{\pgfkeysvalueof{/tikz/arrow inside/end}}
		}
	},
	arrow inside/.style 2 args={
		set arrow inside={#1},
		postaction={
			decorate,decoration={
				markings,Mark/.list={#2}
			}
		}
	},
}

\begin{document}
	\maketitle
	%The assignment is not yet finished unfortunately.
	\tableofcontents
	\section{Semi-Orthogonality and strategic reorthogonalization}
	The Lánczos algortihm can be used to construct a Hessenberg decomposition of a matrix plus a rank 1 term. If we restrict the matrix inputs to real symmetric matrices, then the symmetry requirement will force the apprximately similar Hessenberg matrix to be symmetric as well, which results necessarily in a tridiagonal matrix.
	\begin{algorithm2e}[ht]\label{alg:Lanczos}
		\SetKwInOut{Input}{input}
		\SetKwInOut{Output}{output}
		\SetKw{Init}{init}{}{}
		\SetAlgoLined
		\Input{Linear, symmetric real operator $A$ on $\mathbb{R}$}
		\Output{Approximately orthogonally similar tridiagonal $T\sim A$, given by diagonals $\mathbf{\alpha},\mathbf{\beta}$}
		\Init{choose starting vector $\mathbf{r}_0$,$\beta_0:=\|\mathbf{r}_0\|$, $\mathbf{q}_0=0$}\\
		\For{$j=1,2,...$}{
			$\mathbf{q}_j:=\mathbf{r}_{j-1}/\beta_{j-1}$\\
			$\mathbf{v}_j=A\mathbf{q}_j-\beta_{j-1}\mathbf{q}_{j-1}$\\
			$\alpha_j=\mathbf{v}_j^{\ast}\mathbf{q}_j$\\
			$\mathbf{r}_j=\mathbf{v}_j-\alpha_j\mathbf{q}_j$\\
			$\beta_j=\|\mathbf{r}_j\|$\\
		}
		\caption{Lanczos in exact arithmetic}
	\end{algorithm2e}\\
	
	The matrix $T$ then has the form 
	$$T=\begin{pmatrix}
		\alpha_1&\beta_1&0&\cdots&0\\
		\beta_1&\alpha_2&\beta_2&&0\\
		\vdots&\ddots&\ddots&\ddots&\vdots\\
		0&&\beta_{j-2}&\alpha_{j-1}&\beta_{j-1}\\
		0&\cdots&0&\beta_{j-1}&\alpha_{j}\\
	\end{pmatrix}$$
	One step in the main loop of algorithm \ref{alg:Lanczos} is called a Lanczos step, and is usually written as
	\begin{equation}\label{eq:LanczosStep}
		\beta_j\mathbf{q}_{j+1}=A\mathbf{q}_j-\alpha_j\mathbf{q}_j-\beta_{j-1}\mathbf{q}_{j-1}.
	\end{equation}
	As you know, with $Q_{j}:=[\mathbf{q}_1,\ldots,\mathbf{q}_j]$, we have the relation
	\begin{equation}\label{eq:matLanczos}
		AQ_{j}=Q_jT_j+\beta_{j}\mathbf{q}_{j+1}\mathbf{e}^{\ast}.
	\end{equation}
	In exact arithmetic the above is a great procedure for tridiagonalization. Numerically this is
	not the case unfortunately, as the orthogonality of $Q$ gets diminished by numerical errors.

\FloatBarrier
\newpage
\printbibliography
\end{document}