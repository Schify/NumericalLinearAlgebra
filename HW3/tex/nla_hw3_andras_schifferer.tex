\documentclass{article}
\usepackage[top=2cm, bottom=2cm, left=2.2cm, right=2.5cm]{geometry}
\usepackage{amsmath,amsfonts,amssymb,amsthm}
\usepackage{enumerate}% http://ctan.org/pkg/enumerate

\usepackage[]{pdfcomment}
%As recommended by matlab2tikz
\usepackage{pgfplots}
\usepackage{tikz}
\usepackage{nicematrix}
\usepackage{calc}
\usepackage{pgfplotstable}
\usepackage{booktabs}
\usepackage{csvsimple}
\usepackage{etoolbox}
\usepackage[style=alphabetic]{biblatex}
\usepackage{amsthm}
\bibliography{refs.bib}
%% the following commands are needed for some matlab2tikz features
\usetikzlibrary{plotmarks}
\usetikzlibrary{arrows.meta}
\usepgfplotslibrary{patchplots}
\usepackage{grffile}
%% you may also want the following commands
\pgfplotsset{plot coordinates/math parser=false} 
\newlength\figureheight
\newlength\figurewidth 
%\newlength\figureheight
%\newlength\figurewidth
\usepackage{hyperref}
\usepackage{placeins,nicefrac}
\usepackage{xstring}
\usepackage[]{pgfkeys}
\usepackage{xfrac}
\usepackage{breqn}%dmath, might be problematic sometimes
\usepackage[newfloat]{minted}
\usetikzlibrary{external}
\usetikzlibrary{calc,math}
\tikzexternalize[prefix=extern/]
\usepgfplotslibrary{groupplots}
\usepgfplotslibrary{colormaps}
\usetikzlibrary{decorations.markings,decorations.pathreplacing,patterns,patterns.meta,pgfplots.fillbetween}
\usetikzlibrary{plotmarks}
\usepackage[ruled,vlined,linesnumbered,algo2e]{algorithm2e}
\usepackage{wrapfig}
\usepackage{subcaption}
\newenvironment{code}{\captionsetup{type=listing}}{}
\SetupFloatingEnvironment{listing}{name=Source Code}
\graphicspath{../plots/}

\usepackage{wasysym}
\usepackage{animate}
\usepackage[symbol]{footmisc}

%\usepackage{expl3}
%\ExplSyntaxOn
%\cs_set_eq:NN \fpeval \fp_eval:n
%\ExplSyntaxOff

\usetikzlibrary{fpu}
\graphicspath{{../figs/}}
\newcommand{\figurescale}{1.0}

\newcommand\minput[1]{%
	\input{#1}%
	\ifhmode\ifnum\lastnodetype=11 \unskip\fi\fi}
%\usepackage{titlesec}
%\titleformat{\subsection}{\normalfont\large\bfseries}{Task \thesubsection}{1em}{}
%\titleformat{\section}{\normalfont\Large\bfseries}{Assignment \thesection}{1em}{}
%\titleformat{\subsubsection}{\normalfont\bfseries}{Question \thesubsubsection}{1em}{}
\newcommand\mycommfont[1]{\ttfamily\textcolor{blue}{#1}}
\SetCommentSty{mycommfont}

\newcommand{\stilltodo}[1]{{\color{red} UNFINISHED #1}}

\makeatletter
\newcommand{\trp}{%
	{\mathpalette\@transpose{}}%
}
\newcommand*{\@transpose}[2]{%
	% #1: math style
	% #2: unused
	\raisebox{\depth}{$\m@th#1\intercal$}%
}

\makeatother
\newcommand{\diff}{\mathrm{d}}
\newcommand{\mbf}[1]{\mathbf{#1}}
\author{Andr\'as Schifferer, r0915705}
\title{Numerical Linear Algebra [H03G1A]\\{\LARGE HW3}\\{\large Model Order Reduction}}
\date{\today} 

\pgfplotsset{plot coordinates/math parser=false}

\makeatletter
\providecommand{\leftsquigarrow}{%
	\mathrel{\mathpalette\reflect@squig\relax}%
}
\newcommand{\reflect@squig}[2]{%
	\reflectbox{$\m@th#1\rightsquigarrow$}%
}
\makeatother


\newtheorem{lemma}{Lemma}
\newcommand{\lemmaautorefname}{Lemma}
\newtheorem{theorem}{Theorem}
\DeclareMathOperator*{\argmin}{arg\,min} % thin space, limits underneath in displays
\DeclareMathOperator*{\argmax}{arg\,max} % thin space, limits underneath in displays

\begin{document}
	\maketitle
	%The assignment is not yet finished unfortunately.
	\tableofcontents
	
	% %%%%%%%%%%%%%%%%%%%
	\section{Theory}
	% %%%%%%%%%%%%%%%%%%%
	In this assignment we will be analyzing large linear systems arising from a variety of applications, and the ways in which we can reduce the complexity of such systems. This can be seen as reducing the order of the models involved hence \emph{Model Order Reduction}. In particular, we will look at three types of large state space models.
	\subsection{Basic state space models}
	Given a simple ODE
	\begin{equation}\label{eq:eqFirstOrder}
		\dot{\mathbf{x}}=A\mathbf{x}+\mathbf{b}\cdot u(t)
	\end{equation}
	with solution $\mathbf{q}(t)\in \mathbb{R}^n$, and some input $u(t)\in\mathbb{R}$ we can transform this using the Laplace transform and slightly abusing notation (recycling the variable $\mathbf{x}$) into
	$$s\mathbf{x}-A\mathbf{x}=\mathbf{b}u.$$
	We are not always interested in $\mathbf{x}$, but rather some functional applied to $\mathbf{x}$, which results in the state space model
	\begin{equation}\label{eq:stateSpace}
		\begin{aligned}
			s\mathbf{x}-A\mathbf{x}&=\mathbf{b}u\\
			y&=\mathbf{c}^T\mathbf{x}.
		\end{aligned} 
	\end{equation}
	State space models of this kind will be referred to in this text as \emph{basic state space models}.
	
	
	\subsection{State space models from mechanics}
	Typical ODEs for mechanical systems take the form
	\begin{equation}\label{eq:SecondOrder}
		M\ddot{\mathbf{x}}+D\dot{\mathbf{x}}+K\mathbf{x}=\mathbf{b} \cdot u(t).
	\end{equation}
	Here $\mathbf{x}(t)\in\mathbb{R}^n$ and $\mathbf{b}\cdot u(t)\in\mathbb{R}^{n}$ are vectors varying over time (note the special structure of the time variation of the right hand side!). Typically, $\mathbf{x}$ is a vector of displacements of degrees-of-freedom (dofs) arising from a Finite Element Method (FEM) discretization of the original continuous problem. The vector $\mathbf{b} \cdot u(t)$ then typically corresponds to a force exerted on the structure and is called the \emph{input}. The matrices $M$, $D$ and $K$ are called the \emph{mass}-, \emph{damping}- and \emph{stiffness matrix} respectively.
	
	
	%\textbf{Transform equation \ref{eq:SecondOrder} into an equivalent ODE of the form}
	We can transform the 2nd order ODE into a 1st order one by introducing a new variable $\textbf{q} = \begin{bmatrix}
		\textbf{x}^T & \dot{\textbf{x}}^T
	\end{bmatrix}^T \in \mathbb{R}^{2n}$. Then we can write the \autoref{eq:SecondOrder} in a blocked form as 
	\begin{equation}\label{eq:equivFirstOrder}
		\dot{\mathbf{q}} = \begin{bmatrix}
			\dot{\textbf{x}} \\\hline \ddot{\textbf{x}}
		\end{bmatrix} =
		\underbrace{\left[\begin{array}{@{}c|c@{}}
		\textbf{0}_{n\times n} & \textbf{I} \\\hline -{M}^{-1} K & -{M}^{-1} D
		\end{array}\right]}_{=A} 
		\begin{bmatrix}
			{\textbf{x}} \\\hline \dot{\textbf{x}}
		\end{bmatrix}
		 + \underbrace{\left[\begin{array}{@{}c@{}}
		\textbf{0}_{n\times 1}  \\\hline {M}^{-1} \textbf{b}
		\end{array}\right]}_{\textbf{f}} u(t)
		= A\mathbf{q}+\mathbf{f}\cdot u(t)
	\end{equation}
	where we assumed that $\textbf{M}$ is invertible.
%	\textbf{with $\mathbf{q}(t)\in\mathbb{R}^{2n}$. NOTE THE CHANGE IN DIMENSION. Use the most natural transformation. You can assume that $M$ is invertible.}
		Again, using the Laplace transform and a slight abuse of notation we re-write equation this as
	\begin{equation}\label{eq:LaplaceX}
		s\mathbf{x}-A\mathbf{x}=\mathbf{b}u
	\end{equation} 
	Note that after the two transformations introduced before, these variable are linked, but not at all identical to their original counterparts.
	
	%\textbf{What if we generalize, and instead look for a Laplace domain system of the form}
	We can be more general in our treatment if instead of assuming invertibility of $M$, we just have 
	\begin{equation}\label{eq:equivFirstOrder-generalized}
		E \dot{\mathbf{q}} = \underbrace{\left[\begin{array}{@{}c|c@{}}
				\textbf{I} & \textbf{0}_{n\times n} \\\hline \textbf{0}_{n\times n} & M
			\end{array}\right]}_{E} \begin{bmatrix}
			\dot{\textbf{x}} \\\hline \ddot{\textbf{x}}
		\end{bmatrix} =
		\underbrace{\left[\begin{array}{@{}c|c@{}}
				\textbf{0}_{n\times n} & \textbf{I} \\\hline - K & - D
			\end{array}\right]}_{=A} 
		\begin{bmatrix}
			{\textbf{x}} \\\hline \dot{\textbf{x}}
		\end{bmatrix}
		+ \underbrace{\left[\begin{array}{@{}c@{}}
				\textbf{0}_{n\times 1}  \\\hline  \textbf{b}
			\end{array}\right]}_{\textbf{f}} u(t)
		= A\mathbf{q}+\mathbf{f}\cdot u(t)
	\end{equation}
	which results in 
		\begin{equation}\label{eq:LaplaceXE}
			sE\mathbf{x}-A\mathbf{x}=\mathbf{b}u
		\end{equation}
		after Laplace transformation (with abuse of variable notation again). Since this does not require that $M$ be full rank, it is a slightly more general formulation, even though it can be more difficult to handle.
%		\textbf{with $E\in\mathbb{R}^{2n}$ not necessarily the identity? What is then the most natural such formulation? What important advantage does it have? Show also that, under the corresponding assumptions, each of the above models is equivalent to the original model.}\\
	Typically, we are not interested in all of $x$, but only in some (scalar) function $y$ of $\mathbf{x}$. This is written as
	\begin{equation}\label{eq:stateSpace-again}
		\begin{aligned}
			s\mathbf{x}-A\mathbf{x}&=\mathbf{b}u\\
			y&=\mathbf{c}^T\mathbf{x}.
		\end{aligned} 
	\end{equation}
	\autoref{eq:stateSpace-again} is referred to as the \emph{state space} model of the mechanical system. %\textbf{Show that, under the assumptions made earlier about this model, it is equivalent to the model}
	We can use our definitions in \autoref{eq:equivFirstOrder}, to substitute into \autoref{eq:stateSpace-again} the definition of $A$,$\textbf{x}=\textbf{q}$ and $\textbf{b}$. Then we get the following equation.
	\begin{equation}
		\begin{bmatrix}
			s\textbf{x} \\\hline s^2 \textbf{x}
		\end{bmatrix} - 
		\underbrace{\left[\begin{array}{@{}c|c@{}}
				\textbf{0}_{n\times n} & \textbf{I} \\\hline -{M}^{-1}K & -{M}^{-1} D
			\end{array}\right]}_{=A} 
		\begin{bmatrix}
			{\textbf{x}} \\\hline s{\textbf{x}}
		\end{bmatrix}
		= \underbrace{\left[\begin{array}{@{}c@{}}
				\textbf{0}_{n\times 1}  \\\hline {M}^{-1} \textbf{b}
			\end{array}\right]}_{\textbf{f}=\textbf{b}} u
	\end{equation}
	which is esentially \autoref{eq:equivFirstOrder}'s Laplace transform. Taking the bottom half of the equation, multiplying with $M$ and rearranging one can then obtain
		\begin{equation}\label{eq:Quadratic}
			\begin{aligned}
				s^2M\mathbf{x}+sD\mathbf{x}+K\mathbf{x}&=\mathbf{b}u\\
				y&=\mathbf{c}^T\mathbf{x}.
			\end{aligned} 
		\end{equation}
		{with $x\in\mathbf{R}^n$.} The output equation is unchanged due to the linearity of the Laplace transform. 
		The model above is referred to as the \emph{quadratic model} of the mechanical system (it is the Laplace transform of \autoref{eq:SecondOrder}).\\
		
	The most important data associated to a state space model is the so-called \emph{transfer function} $H(s):=y(s)/u(s)$.
%	\textbf{Interpret the meaning of the transfer function. Show that it can be written as}
	This relates the input and the output of the system within the Laplace domain. In particular, it is the Laplace transform of the time zero impulse response of the system, where $u(t) = \delta(t) = \mathcal{L}^{-1}\left\{1\right\}$, with $\delta(t)$ being the Dirac delta function. Based on this one information, the response of the system can be calculated to any arbitrary input.
	
	We can form the transfer function from \autoref{eq:stateSpace-again} as 
	\begin{equation}
		H(s) = \frac{y(s)}{u(s)} = \frac{\textbf{c}^T \textbf{x}}{u(s)} = \frac{\textbf{c}^T \left(s I - A\right)^{-1}\textbf{b} u(s) }{u(s)} = \textbf{c}^T \left(s I - A\right)^{-1}\textbf{b}
	\end{equation}
	If we suppose that we can form the eigendecomposition of the state matrix as $A = P \Lambda P^{-1} = P \Lambda Q^\ast$, then this can be further understood as 
	\begin{equation}\label{eq:transferFunction}
		\begin{aligned}
			H(s)&=\mathbf{c}^T(sI-A)^{-1}\mathbf{b} 
			= \mathbf{c}^T(sP P ^{-1}-P \Lambda P^{-1})^{-1}\mathbf{b}
			= \mathbf{c}^T\left(P \left(s I -\Lambda \right)P^{-1}\right)^{-1}\mathbf{b} 
			= \mathbf{c}^T P \left(s I -\Lambda \right)^{-1}P^{-1}\mathbf{b}  \\
			&
			= \begin{bmatrix}
				\textbf{c}^T \textbf{p}_1 & \textbf{c}^T \textbf{p}_2 & \dots & \textbf{c}^T \textbf{p}_n
			\end{bmatrix}
			\begin{bmatrix}
				s - \lambda_1 & & & \\
				& s - \lambda_2 & & \\
				&& \ddots & \\
				&&& s - \lambda_n
			\end{bmatrix}^{-1}
			\begin{bmatrix}
				\textbf{q}_1^\ast \textbf{b} \\ \textbf{q}_2^\ast \textbf{b} \\ \vdots \\ \textbf{q}_n^\ast \textbf{b}
			\end{bmatrix}
			=\sum_{i=1}^n \frac{(\mathbf{c}^T\mathbf{p}_i)(\mathbf{q}_i^{\ast}\mathbf{b})}{s-\lambda_i}
		\end{aligned} 
	\end{equation}
	following from the orthogonality of $\Lambda$. $\textbf{p}_i$ and $\textbf{q}_i$ are the columns of $P$ and $Q$ respectively.
	It clearly shows that $H(s)$ is a rational function of degree $n$ (note that this $n$ is again a victim of notation abuse), with 
	This is equivalent to the statement
	 show that
	\begin{equation}
		(sI-A)^{-1}=\sum_{i=1}^{n}\frac{\mathbf{p}_i\mathbf{q}_i^{\ast}}{s-\lambda_i}.
	\end{equation}
	
	he second form of equation \ref{eq:transferFunction} we will call the \emph{residual form} of the transfer function.
	
	%\textbf{What does $H$ look like in terms of the quadratic model? Describe the advantages of this formulation over the above (theoretical) descriptions.}
	If we express $H(s)$ directly from \autoref{eq:Quadratic}, we get an alternative representation of the form:
	\begin{equation}
		H(s) = \frac{y(s)}{u(s)} = \frac{\textbf{c}^T \textbf{x}}{u(s)}
		= \frac{\textbf{c}^T \left(M s^2 + D s + K\right)^{-1} \textbf{b}u(s)}{u(s)} = \textbf{c}^T \left(M s^2 + D s + K\right)^{-1} \textbf{b},
	\end{equation}
	where the inverse of only an $n \times n$ matrix is taken. This again does not rely on the inversion of the matrix $M$, so it is an alternative formulation to that described before.
	
	\subsection{State space models from simple ODEs}
	Naturally we can consider any linear ODE to obtain a state space model i.e.
	\begin{equation}\label{eq:simpleODE}
		\begin{aligned}
			E\dot{\mathbf{x}}(t)&=A\mathbf{x}(t)+\mathbf{b}\cdot u(t)\\
			y(t)&=\mathbf{c}^{T}\mathbf{x}(t),
		\end{aligned}
	\end{equation}
	with $\mathbf{x},\mathbf{b},\mathbf{c}\in\mathbb{R}^{n}$ and $E,A\in\mathbb{R}^{n\times n}$ leading by way of the Laplace transform to the model
	\begin{equation}\label{eq:SSmodelfromSimpleODE}
		\begin{aligned}
			sE\mathbf{x}&=A\mathbf{x}+\mathbf{b}u\\
			y&=\mathbf{c}^{T}\mathbf{x}.
		\end{aligned}
	\end{equation}
	Note again the abuse in notation, and that in general $E$ is not the identity.
%	\textbf{ What does the transfer function look like now? Can it still be written in some residual form as in equation \ref{eq:transferFunction}? If so, under what assumptions? HINT: this requires looking into the generalized eigenvalue decomposition.}
	Here, analogous to the previous derivation without $E$, the transfer function can be expressed as 
	\begin{equation}\label{eq:Hs-matrix-gen}
		H(s) = \textbf{c}^T \left(sE - A\right)^{-1} \textbf{b}.
	\end{equation}
	If we assume, that $(A, E)$ have a generalized eigenvalue decomposition of the form $A = E P \Lambda P^{-1}$, then we can make the following transformations
	\begin{equation}
		\left(sE - A\right)^{-1} = \left(sEP P^{-1}-EP\Lambda P^{-1}\right)^{-1} = \left(EP\left(s I -\Lambda\right) P^{-1}\right)^{-1} = P \left(s I -\Lambda\right)^{-1} P^{-1} E^{-1}
	\end{equation}
	with $\Lambda$ being the diagonal vector of the generalized eigenvalues. If we denote $\tilde{Q}^\ast := P^{-1}E^{-1} $ similarly to before, then 
	\begin{equation}
		(sI-A)^{-1}=\sum_{i=1}^{n}\frac{\mathbf{p}_i\tilde{\mathbf{q}}_i^{\ast}}{s-\lambda_i}.
	\end{equation}
	and by extension
	\begin{equation}
		H(s)
		=\sum_{i=1}^n \frac{(\mathbf{c}^T\mathbf{p}_i)(\tilde{\mathbf{q}}_i^{\ast}\mathbf{b})}{s-\lambda_i}
	\end{equation}1
	will also hold in this case.
	
	% %%%%%%%%%%%%%%%%%%%
	\section{Applications}
	% %%%%%%%%%%%%%%%%%%%
	In this section three applications are outlined. We have been given code \texttt{bode\_from\_system} and \texttt{bode\_from\_function} that can produce bode plots for transfer functions given in descriptor form or in the form of a transfer function definition, as well as a script that produces these plots for the relevant frequency ranges.
	These tools have been used in the creation of the subsequent figures.
	 %\textbf{Study the systems and these functions carefully.}
	
	\vspace{\baselineskip}
	
%	\textbf{All matrices and vectors have been provided in Matrix Market format or \texttt{.mat} format. In addition you have been provided a file called `\texttt{mmread.m}' which contains a function for reading in these matrices in MatLab.	}
	
	
	\subsection{Spiral inductor}
	We look here at an integrated RF passive inductor, of the type `spiral inductor'. Spiral inductors, which vary in structure, are among the most common types of on-chip inductors. Spiral inductors are usually characterized by the diameter, the line width, the number of turns, and the line-to-line space. In this case, the inductor has turns that are 40$\mu$m wide, 15$\mu$m thick, with a separation of 40$\mu$m. The spiral is suspended 55$\mu$m over a substrate by posts at the corners and centers of the turns in order to reduce the capacitance to the substrate. To make it also a proximity sensor, a 0.1$\mu$m plane of copper is added 45$\mu$m above the copper spiral.
	The overall extent of the suspended turns is 1.58mm × 1.58mm. The spiral inductor, including
	part of the overhanging copper plane, is shown in figure \ref{fig:PEEC}. 
	\begin{figure}[h]
		\center
		\includegraphics[width=.45\linewidth]{images/spiral_inductor.png}
		\caption{Spiral inductor}\label{fig:PEEC}
	\end{figure}
	The model is discretized using a FEM-like technique called PEEC, which results in mesh specific inductance and resistance matrices $L$ and $R$ respectively, corresponding to the following differential state-space form:
	\begin{align*}
		L\frac{di_m}{dt}&=Ri_m+Nv_p\\
		i_p&=N^Ti_m
	\end{align*}
	Here $i_m$ is the mesh current and $v_p$ and $i_p$ are the voltage and current at nodes of interest, with $N$ a `natural' matrix mapping between these nodes and the mesh.
	%\textbf{You do not need to understand this in detail! What does the associated Laplace domain state space model look like?}
	The associated Laplace domain state space model will be
	\begin{equation}
		\begin{aligned}
			sL i_m - R i_m &= N v_p\\
			i_p&=N^Ti_m
		\end{aligned}
	\end{equation} 
	In the multi-gigahertz frequency range, the so-called `skin effect' causes current to
	flow only at the surface of conductors (i.e. the copper coil and plane), leading to a decrease of wire
	inductance and an increase of resistance. Capturing the skin effect while also maintaining an accurate low frequency response is a challenge for many model reduction algorithms. For that reason the frequency range considered for this model is very wide: $\omega\in[1,10^{10}]$.
	%\textbf{You have been given a MatLab file containing all the necessary data for this system. Analyze the system in detail and report your findings. Use the function \texttt{bode\_from\_system} or \texttt{bode\_from\_function} to evaluate and plot transfer function in the frequency range $\omega\in [1,10^{10}] $. Recall that $s=2\pi\imath\omega$.}
	
	Using the theory we derived earlier and the provided matlab files, we can plot the Bode diagram of the system as in \autoref{fig:bode-spiral}
		\begin{figure} [h!]
		\centering
		\setlength{\figurewidth}{2\textwidth}
		\tikzsetfigurename{bode_sprial_inductor}
		\input{figs/bode_sprial_inductor.tex}
		\caption{Bode plot for spiral inductor}\label{fig:bode-spiral}
	\end{figure}
	\subsection{Butterfly Gyroscope}
	In inertial navigation, rotation sensors, also called gyroscopes, are indispensible. One such sensor, at the high-end of the quality range is the Butterfly gyroscope. You can see a diagram of this gyro in figure \ref{fig:butterFly2}
	\begin{figure}[h]
		\center
		\includegraphics[width=.5\linewidth]{images/butterfly.png}
		\caption{Butterfly Gyroscope}\label{fig:butterFly2}
	\end{figure}
	By applying DC-biased AC-voltages, the Butterfly wings are forced to vibrate in anti-phase in the wafer plane. This vibrational mode is called the excitation mode. As the structure containing the gyro rotates about the axis of sensitivity (see Figure \ref{fig:butterFly2}), each of the masses will be affected by a Coriolis acceleration. The Coriolis force induces an anti-phase motion of the wings out of the wafer plane, which can be measured via different electrodes. The external angular velocity can be related to the amplitude of this vibrational mode. Since this gyro must be fast and precise, it should be clear to you that a high-precision, but efficiently computable structural model of the gyro must be implemented.
	
	The Gyro is discretized using FEM, resulting in a model of the form
	\begin{equation}\label{eq:gyromodel}
		\begin{aligned}
			M\ddot{\mathbf{x}}+\beta K\dot{\mathbf{x}}+K\mathbf{x}&=B \cdot u(t).\\
			y(t)&=C^T \mathbf{x}(t)
		\end{aligned}
	\end{equation}\\
	Using the theory we derived earlier and the provided matlab files, we can plot the Bode diagram of the system as in \autoref{fig:bode-gyro}.
%	\textbf{Note that multiple input and output states are considered! You do not have to report on all input-output pairs, but your methods should work for each of them! You have been given a MatLab file containing all the necessary data for this system. Take $\beta=1e-6$ Analyze the system in detail and report your findings. Use the function \texttt{bode\_from\_system} or \texttt{bode\_from\_function} to evaluate and plot transfer function in the frequency range $\omega\in [1,10000] $. Recall that $s=2\pi\imath\omega$.}
	
	\subsection{International space station component}
	Control is another important source of model order reduction, since the complexity of a controller is dominated by the complexity of the corresponding system. In the International Space Station (ISS), assembly, testing and maneuvring is done in `stages' i.e. parts of the ISS. One such stage, stage 12A, is depicted in figure \ref{fig:stage12A}.
	\begin{figure}[h]
		\center
		\includegraphics[width=.5\linewidth]{images/iss.png}\label{fig:stage12A}
		\caption{Stage 12A of the ISS. Source: \emph{First Ever Flight Demonstration of Zero Propellant Maneuver(TM) Attitude Control Concept}, Bedrossian, N. \& Bhatt, S.,2007}
	\end{figure}
	This stage leads to a large sparse system to be reduced in order. This system was determined using techniques from standard systems theory, and is of the form
	\begin{align*}
		s\mathbf{x}-A\mathbf{x}&=B\\
		\mathbf{y}&=C^T\mathbf{x}
	\end{align*}
%	\textbf{Note that multiple input and output states are considered! You do not have to report on all input-output pairs, but your methods should work for each of them!}
	Due to the large size of stage 12A and the absense of oscillatory air forces in space, we are mainly interested in the low-frequency behavior.
%	\textbf{You have been given a MatLab file containing all the necessary data for this system. Analyze the system in detail and report your findings. Use the function \texttt{bode\_from\_system} or \texttt{bode\_from\_function} to evaluate and plot transfer function in the frequency range $\omega\in [10^{-2},10^2]$. Recall that $s=2\pi\imath\omega$.}
	In \autoref{fig:bode-iss1}, \ref{fig:bode-iss2}, and \ref{fig:bode-iss3} we have plotted all of the possible transfer functions of the system. We observe lots of poles in the $\left[10^{-1}, 10^{1}\right]$ range, foreshadowing difficulties in the accurate approximation of the system.
	\begin{figure}[h!]
		\centering
		\setlength{\figurewidth}{2\textwidth}
		\tikzsetfigurename{bode_butterfly_gyro}
		\input{figs/bode_butterfly_gyro.tex}
		\caption{Bode plot for butterfly gyroscope}\label{fig:bode-gyro}
	\end{figure}
	
	
	\begin{figure}[h!]
		\centering
		\setlength{\figurewidth}{2\textwidth}
		\tikzsetfigurename{bode_iss_i1}
		\input{figs/bode_iss_i1.tex}
		\caption{ISS A12 - Input Channel 1}\label{fig:bode-iss1}
	\end{figure}
	\begin{figure}[h!]
		\centering
		\setlength{\figurewidth}{2\textwidth}
		\tikzsetfigurename{bode_iss_i2}
		\input{figs/bode_iss_i2.tex}
		\caption{ISS A12 - Input Channel 2}\label{fig:bode-iss2}
	\end{figure}
	\begin{figure}[h!]
		\centering
		\setlength{\figurewidth}{2\textwidth}
		\tikzsetfigurename{bode_iss_i3}
		\input{figs/bode_iss_i3.tex}
		\caption{ISS A12 - Input Channel 3}\label{fig:bode-iss3}
	\end{figure}
	
	% %%%%%%%%%%%%%%%%%%%
	\FloatBarrier
	\section{Approximations of the transfer function}
	% %%%%%%%%%%%%%%%%%%%
	Now that we know our models, we can start reducing them. This means approximating the actual transfer function by some approximant $\hat{H}$. In this section you are asked to construct such an $\hat{H}$ in three of ways. Keep the following questions in mind: in what way is the transfer function approximated? \\
	The best way to compare transfer functions is to look at the $\mathcal{H}_2$ norm difference of the transfer functions in the operationally relevant regimes. Due to a lack of time no rigorous comparison could be performed.
	%\textbf{Compare and contrast the properties of these approximizations, the (typical) quality of the approximation reached, the overall performance,...}\\ 
	\subsection{Dominant pole methods}
	\subsubsection*{DPA}
	In class we have seen modal truncation as a method for model order reduction. One example of this is the \emph{Dominant Pole Algorithm} (DPA). The general form of DPA is given in algorithm \ref{alg:DPA}.
	\begin{algorithm2e}[ht]
		\SetKwInOut{Input}{input}
		\SetKwInOut{Output}{output}
		\SetKw{Init}{init}{}{}
		\SetAlgoLined
		\Input{System $(E, A, b, c)$, initial pole estimate $s_0$ , tolerance $\epsilon$}
		\Output{Approximate dominant pole $\hat{\lambda}$ and corresponding eigenpair $(\mathbf{x},\mathbf{y})$}
		\Init{$k:=0$, err = $\infty$}\\
		\While{$\text{err}>\text{tol}$}{
			Solve $(s_kE-A)\mathbf{v}_k = \mathbf{b}$\\
			Solve $(s_kE-A)^{\ast}\mathbf{w}_k = \mathbf{c}$\\
			Compute the new pole estimate\\
			$s_{k+1} = s_k -\frac{\mathbf{c}^{\ast} \mathbf{v}_k}{\mathbf{w}_k^{\ast} E\mathbf{v}_k}
			=\frac{\mathbf{w}_k^{\ast}A\mathbf{v}_k}{\mathbf{w}_k^{\ast}E\mathbf{v}_k}$\\
			$\mathbf{x}:=\mathbf{v}_k/\|\mathbf{v}_k\|$\\
			$\mathbf{y}:=\mathbf{w}_k/\|\mathbf{w}_k\|$\\
			$\text{err}:=\|A\mathbf{x} - s_{k+1}E\mathbf{x}\|_2$\\
			$k := k + 1$
		}
		\caption{DPA}
		\label{alg:DPA}
	\end{algorithm2e}
	
	\textbf{Explain DPA and its relation to Newton's method. Implement it in \texttt{Matlab}}\\
	
	DPA can be understood as Newton's algorithm applied to find a pole of $H$. As a pole of $H$, is a root of the denominator of $H$ \footnote{we have seen before how $H(s)$ can be represented as a rational fraction for simple LTI (Linear Time Invariant) systems}, let us define $Q(s) = \left[H(s)\right]^{-1}$. Furthermore, derivating this expression, we would end up with $Q^\prime(s) = -\left[H(s)\right]^{-1} H^\prime(s) \left[H(s)\right]^{-1}$ (see \autoref{lem:inv_derivative}).
	
	To find a root of $Q$, Newton's iteration becomes:
	\begin{equation}\label{eq:newton-it-Q2H}
		s_{k+1} = s_{k} - \left[Q^\prime(s)\right]^{-1} Q(s) = s_{k} - \left[-\left[H(s)\right]^{-1} H^\prime(s) \left[H(s)\right]^{-1}\right]^{-1} \left[H(s)\right]^{-1} = s_{k} + \left[H^\prime(s)\right]^{-1} H(s)
	\end{equation} 
	
	Since we know the formula for the transfer function of the generalized system (see \autoref{eq:Hs-matrix-gen}), we can also calculate its derivative as:
	\begin{equation}
		H'(s) = \textbf{c}^T \left(sE - A\right)^{-1} E \left(sE - A\right)^{-1} \textbf{b}.
	\end{equation}
	As was denoted during the lecture, for a given value of $s$, the regular and adjunct state vector is 
	\begin{align}
		\textbf{x} &= \left(sE - A\right)^{-1} \textbf{b},\\
		\textbf{y}^T &= \textbf{c}^T \left(sE - A\right)^{-1}
	\end{align}
	which means that 
	\begin{equation}
		H'(s) = -\textbf{y}^\ast  E \textbf{x}.
	\end{equation}
	This is quite similar in calculation to $H(s) = \textbf{y}^T\textbf{b} = \textbf{c}^T \textbf{x}$, which are all easy to implement. The Newton iteration therefore further simplifies to 
	\begin{equation}
		s_{k+1}  = s_{k} - \left[\textbf{y}^\ast E \textbf{x}\right]^{-1} \textbf{c}^T \textbf{x}.
	\end{equation}
	Using the fact that $\textbf{c}^T =^T \textbf{y}^\ast \left(sE - A\right)  \leftrightarrow s \textbf{y}^\ast E - \textbf{c} = \textbf{y}^T A$, we can reformulate this previous as 
	\begin{equation}
		s_{k+1}  = s_{k} - \left[\textbf{y}^\ast E \textbf{x}\right]^{-1} \textbf{c}^T \textbf{x} b = \left[\textbf{y}^T E \textbf{x}\right]^{-1} \left(s_k \textbf{y}^T E - \textbf{c}^T\right)\textbf{x} = \left[\textbf{y}^T E \textbf{x}\right]^{-1} \textbf{y}^T A\textbf{x}.
	\end{equation}
	Note that the notation in \autoref{alg:DPA} is slightly different, than that used in the derivation, due to the normalization of the eigenvector candidates. 
	
	When applying the algorithm, we take an initial value from the right half of the complex plane, so that in the case of a stable system we would converge to the pole with the smallest distance from the imaginary axis (usually the dominant pole).
	
	DPA suffers from multiple problems:
	\begin{itemize}
		\item It can only be used to approximate one pole.
		\item Convergence can be problematic in some special cases, or if the initial guess is not good enough.
		\item Dominant poles are defined as poles with the largest contribution towards the eigenfunction, i.e. where 
		\begin{equation*}
			\frac{\left|R_i\right|}{\left|\Re\left(\lambda_i\right)\right|}
		\end{equation*}
		is the largest. As DPA will find the eigenvalue, where $\left|\Re\left(\lambda_i\right)\right|$ is the smallest, there is no theoretical guarantee that it will be the dominant pole.
	\end{itemize}
	
	The failure of the dominant pole is illustrated in \autoref{fig:dpa-spiral}. As we can see the estimation is quite bad. While there are sections which do match, most of the approximation is not good, and the DC gain ($\left|H(s)\right|$ as $s \to 0$) seems to be missed as well. 
	
	\begin{figure}[h!]
		\centering
		\setlength{\figurewidth}{2\textwidth}
		\tikzsetfigurename{sprial_inductor_dpa}
		\input{figs/bode_sprial_inductor_dpa_sadpa.tex}
		\caption{\texttt{DPA} (\autoref{alg:DPA}) and \texttt{SADPA} applied to the Spiral Inductor}\label{fig:dpa-spiral}
	\end{figure}
	
	
	\subsection*{SADPA}
%	\textbf{Simple DPA is not a very good or useful algorithm. Explain why. Use this to motivate \emph{subspace accelerated DPA} (SADPA) as you have seen it in the slides.}
	SADPA (Subspace Accelerated Dominant Poles Algorithm) is an algorithm which aims to correct the flaws as listed for DPA. Instead of discarding the state vector estimates in the algorithm, they are used to construct a subspace (via Gram-Schmidt orthogonalization). This allows the algorithm to converge to multiple eigenvalues at the same time and make better educated choices regarding which is dominant out of them. 
	
	The algorithm has been introduced and implemented in the paper by \textcite{sadpa2006rommes}. \texttt{SADPA} usin 20 poles, which it estimated using 141 LU solves. The used parameters can be found in \autoref{table:sadpa-prob1-opts}. With only 20 poles the algorithm was able to successfully converge to a good approximation of the transfer function, as seen \autoref{fig:dpa-spiral}. Note that for higher frequencies this approximation gets relatively worse, as we only care about the absolute error. Still the approximation is close enough even in this range.
	
	\begin{table}[h!]
		\centering
		\begin{tabular}{llll}
use_lu & dpa_bordered & strategy & nwanted \\ 
\hline 
0 & 0 & LM & 20 \\ 
\hline 
\end{tabular}
		\caption{The parameters used for \texttt{SADPA} algorithm for the Spiral Inductor}\label{table:sadpa-prob1-opts}
	\end{table}
	
	The same parameters applied to the Butterfly Gyroscope model with \texttt{SADPA} took too much time to converge, presumably due to the badness of the algorithm, with too many factorizations.
	%You have been provided an algorithm that implements SADPA. This is an extremely complicated algorithm. You do not need to understand the full algorithm, but analyse the settings section of the preamble carefully. For convenience, you can set \texttt{options.use\_lu=0} and \texttt{opt.dpa\_bordered = 0} to start. However the other options require more careful understanding. Play around a bit and pick heuristically good choices. \textbf{You do not have to provide a proof of optimality, but justify your choices.}\\
	\subsection*{SAQDPA}
	We can also write a version of SADPA specifically tailored to the quadratic model of a mechanical system. For this we use lemma \ref{lem:quadraticDPA}.
	\begin{lemma}[Product Rule For Matrix Functions]\label{lem:matr-product-rule}
		Let $R(s) = \begin{bmatrix}
			\textbf{r}_1(s) & \textbf{r}_2(s) & \dots & \textbf{r}_{n}(s)
		\end{bmatrix}^T \in \mathbb{R}^{n\times n} $, with the rows $\textbf{r}_i$ and $C(s) = \begin{bmatrix}
		\textbf{c}_1(s) & \textbf{c}_2(s) & \dots & \textbf{c}_{n}(s)
		\end{bmatrix}^T \in \mathbb{R}^{n\times n} $. Then the derivative of the product can be expressed using the product rule as
		\begin{equation}
			\frac{\diff}{\diff s} \left(R(s)C(s)\right) = \left(R(s)C(s)\right)^\prime = R^\prime(s) C(s) + R(s) C^\prime(s) = \left[\frac{\diff}{\diff s} R(s)\right]C(s) + R(s)\left[\frac{\diff}{\diff s} C(s)\right]
		\end{equation}
	\end{lemma}
	\begin{proof}
		Let us introduce the product matrix as $ P(s) = R(s)C(s) = \left[\textbf{r}_i^T(s) \textbf{c}_j\right]_{(i,j)}$ with the elements as the products of the corresponding row of $R$ and column of $C$. Then within each of these elements due to the linearity of the sum within the dot product between the row and column, we know that (using the regular product rule for differentiation)
		\begin{equation*}
			\frac{\diff}{\diff s} \left(\textbf{r}_i(s)\textbf{c}_j(s)\right) = \textbf{r}_i^\prime(s)\textbf{c}_j(s) + \textbf{r}_i(s)\textbf{c}^\prime_j(s).
		\end{equation*}
		Thereby $P^\prime(s)$ can be written as 
		\begin{equation*}
			P^\prime(s) = \left[\frac{\diff}{\diff s} \left(\textbf{r}_i(s)\textbf{c}_j(s)\right)\right]_{(i,j)} = \left[\textbf{r}_i^\prime(s)\textbf{c}_j(s) + \textbf{r}_i(s)\textbf{c}^\prime_j(s)\right]_{(i,j)} = \left[\textbf{r}_i^\prime(s)\textbf{c}_j(s)\right]_{(i,j)} + \left[ \textbf{r}_i(s)\textbf{c}^\prime_j(s)\right]_{(i,j)} = R^\prime(s) C(s) + R(s) C^\prime(s)
		\end{equation*}
	\end{proof}
	
	\begin{lemma}\label{lem:inv_derivative}
		Suppose $G(s) \in \mathbb{R}^{n\times n}$ is invertible. Then
		\begin{equation}\label{eq:mat-inv-deriv}
			\frac{d}{d s}G^{-1}(s)=-G^{-1}(s)G'(s)G^{-1}(s)
		\end{equation}
	\end{lemma}
	\begin{proof}
		Since the inverse exists, $I_{n\times n} = G^{-1}(s) G(s) $, corresponding to the identity matrix.
		Using \autoref{lem:matr-product-rule}, boths sides of the equation can differentiated
		\begin{equation*}
			0_{n\times n} = \frac{\diff}{\diff s} \left[ G^{-1}(s) G(s) \right] = G^{-1}(s) \frac{\diff}{\diff s} \left[  G(s) \right] + \frac{\diff}{\diff s} \left[ G^{-1}(s) \right]  G(s)
		\end{equation*}
		Multiplying by $G^{-1}(s)$ from the right and rearranging we arrive at the statement in \autoref{eq:mat-inv-deriv}.
	\end{proof}
	
	\begin{lemma}\label{lem:quadraticDPA}
		Suppose $(s^2M+sD+K):=G(s)$ is invertible. Then
		\begin{align*}
			\frac{d}{d s}G^{-1}(s)&=-G^{-1}(s)G'(s)G^{-1}(s)\\
			&=-(s^2M+sD+K)^{-1}(2sM+D)(s^2M+sD+K)^{-1}
		\end{align*}
	\end{lemma}
	\begin{proof}
		This is a direct result of \autoref{lem:inv_derivative}, applied to $G(s)$, where we know $G^\prime(s) = 2 s M + D$. 
	\end{proof}
	
%	\textbf{Prove this lemma. Use it to construct a Newton method for a quadratic model like in the case of (regular) DPA. Call this algorithm QDPA and write down what this algorithm looks like. Implement this in \texttt{Matlab}}\\
The algorithm for QDPA has been implemented in Matlab, however some bug prevents the presentation of results, as it failed ot converge for the Butterfly Gyroscope.
	
	We can again use Newton's algorithm applied to $Q(s) = \left(H(s)\right)^{-1}$ to find poles of the system. In this case we know that the transfer function of the system is 
	\begin{equation}
		H(s) = \textbf{c}^T \left(M s^2 + D s + K\right)^{-1} \textbf{b} = \textbf{c}^T G^{-1}(s) \textbf{b}.
	\end{equation} 
	Then the derivative using the \autoref{lem:quadraticDPA}
	\begin{equation}
		\begin{aligned}
			H^\prime(s) &= -\textbf{c}^T G^{-1}(s)G'(s)G^{-1}(s) \textbf{b} \\ & = -\textbf{c}^T (s^2M+sD+K)^{-1}(2sM+D)(s^2M+sD+K)^{-1} \textbf{b}.
		\end{aligned}
	\end{equation}
	Therefore, the Newton iteration (like in \autoref{eq:newton-it-Q2H})
	\begin{equation}
		\begin{aligned}
			s_{k+1} &= s_{k} + \left[H^\prime(s_k)\right]^{-1} H(s_k) \\ &=  s_{k} - \left[\textbf{c}^T (s_k^2M+s_kD+K)^{-1}(2s_kM+D)(s_k^2M+s_kD+K)^{-1} \textbf{b}\right]^{-1} \textbf{c}^T \left(M s_k^2 + D s_k + K\right)^{-1} \textbf{b}.
		\end{aligned}
	\end{equation}
	If we introduce the notation again as 
	\begin{align}
		\textbf{v}_k &= (s_k^2M+s_kD+K)^{-1} \textbf{b}\\
		\textbf{w}_k &= (s_k^2M+s_kD+K)^{-\ast} \textbf{c}
	\end{align}
	Then the step becomes 
	\begin{equation}
			s_{k+1}  =  s_{k} - \left[\textbf{w}_k^\ast (2s_kM+D)\textbf{v}_k\right]^{-1} \textbf{w}_k^\ast  \textbf{b}.
	\end{equation}
	All of this together can be used to construct the \texttt{QDPA} algorithm as in \autoref{alg:QDPA}. It has been implemented in Matlab, but it did not provide satisfacotry results for the butterfly gyroscope problem either.
	\begin{algorithm2e}[ht]
		\SetKwInOut{Input}{input}
		\SetKwInOut{Output}{output}
		\SetKw{Init}{init}{}{}
		\SetAlgoLined
		\Input{System $(M, D, K, \textbf{b}, \textbf{c})$, initial pole estimate $s_0$ , tolerance $\epsilon$}
		\Output{Approximate dominant pole $\hat{\lambda}$ and corresponding eigenpair $(\mathbf{x},\mathbf{y})$}
		\Init{$k:=0$, err = $\infty$}\\
		\While{$\text{err}>\text{tol}$}{
			Solve $(s_k^2M+s_kD+K)\mathbf{v}_k = \mathbf{b}$\\
			Solve $(s_k^2M+s_kD+K)^{\ast}\mathbf{w}_k = \mathbf{c}$\\
			Compute the new pole estimate\\
			$s_{k+1} = s_k -\frac{\mathbf{c}^{\ast} \mathbf{v}_k}{\mathbf{w}_k^{\ast} (2s_kM+D)\mathbf{v}_k}$\\
			$\mathbf{x}:=\mathbf{v}_k/\|\mathbf{v}_k\|$\\
			$\mathbf{y}:=\mathbf{w}_k/\|\mathbf{w}_k\|$\\
			$\text{err}:=\|\left(s_k^2M+s_kD+K\right)\textbf{x}\|_2$\\
			$k := k + 1$
		}
		\caption{QDPA}
		\label{alg:QDPA}
	\end{algorithm2e}
	
	
	\subsection{Iterative Rational Krylov}
	Suppose we have a system of the form
	\begin{equation}\label{eq:stateSpace}
		\begin{aligned}
			s\mathbf{x}-A\mathbf{x}&=\mathbf{b}u\\
			y&=\mathbf{c}^T\mathbf{x}.
		\end{aligned} 
	\end{equation}
	with associated transfer function
	$$H(s)=\sum_{i=1}^n\frac{R_i}{s-\lambda_i}.$$
	Theoretically, the Iterative Rational Krylov Algorithm (IRKA) aims at minimizing
	$$\|H-\hat{H}\|_{\mathcal{H}_2}:=\left(\int_{\infty}^{\infty}|H(is)-\hat{H}(is)|ds\right)^{\frac{1}{2}}$$
	over the set $\mathcal{V}_d$ of approximants
	$$\hat{H}(s)=\sum_{i=1}^d\frac{\hat{R}_i}{s-\hat{\lambda}_i}$$
	of fixed Macmillan degree $d$.
	This lemma is useful:
	\begin{lemma}\label{lem:H2opt-G}
		If
		$$G(s)=\sum_{i=1}^{n}\frac{R_i}{s-\lambda_i}$$
		then it holds that
		$$\|G\|_{\mathcal{H}_2}^2=\sum_{i=1}^{n}R_iG(-\lambda_i)$$
		under mild technical assumptions that you can assume satisfied.
	\end{lemma}
	
	$\hat{H}$ will be defined as 
	\begin{equation}
		\hat{H}(s)=\sum_{i=1}^d\frac{\hat{R}_i}{s-\hat{\lambda}_i}
	\end{equation}
	with $\hat{R}_i$ approximate residues and $\hat{\lambda}_i$ approximate eigenvalues ($i=1,2, \dots d$). 
	
	Now define the error function 
	$$\mathcal{E}\left(\hat{R}_1, \hat{R}_2, \dots, \hat{R}_d,\hat{\lambda}_1, \hat{\lambda}_2, \dots, \hat{\lambda}_d \right):=\|H(s)-\hat{H}(s)\|_{\mathcal{H}_2}^2$$
	and observe that in the context of minimizing $\mathcal{E}$ over the set of approximate models $\hat{H}$ of fixed Macmillan degree $d$, $\mathcal{E}$ is a function of $2d$ variables. 
	%\textbf{Explain this. Which variables are this specifically? Use this together with the lemma to show that a necessary condition for $\mathcal{E}$ to be minimized over the set of models $\mathcal{V}_d$ is that}
	
	
	In $\mathcal{E}$'s extrema we can use our knowledge from calculus to find necessary conditions the extremity using the gradient condition. This means that the gradient must be zero in an extrema, and therefore\footnote{Note that the notation is a bit sloppy here, for instance $\lambda_i$ both refers to the parameter value where we are evaluating and also the variable with which we are derivating}
	\begin{equation*}
		\left.\frac{\diff}{\diff \hat{R}_j} \mathcal{E}\right|_{(\hat{R}_j, \hat{\lambda}_j)} = 0\quad \mathrm{and} \quad \left.\frac{\diff}{\diff \hat{\lambda}_j} \mathcal{E} \right|_{(\hat{R}_j, \hat{\lambda}_j)} = 0 \quad \quad \mathrm{for} \, j=1,2, \dots d
	\end{equation*}
	With the help of \autoref{lem:H2opt-G} we know that 
	\begin{equation}
		\begin{aligned}
		\mathcal{E}\left(\hat{R}_1, \hat{R}_2, \dots, \hat{R}_d,\hat{\lambda}_1, \hat{\lambda}_2, \dots, \hat{\lambda}_d \right):=\|G(s)\|_{\mathcal{H}_2}^2 &= \sum_{i=1}^{n} R_i G(-\lambda_i) - \sum_{i=1}^{d} \hat{R}_i G(-\hat{\lambda}_i) \\&=
		\sum_{i=1}^{d} \left(R_i  G(-\lambda_i) -\hat{R}_i G(-\hat{\lambda}_i) \right) - \sum_{i=d+1}^{n} {R}_i G(-{\lambda}_i) 
		\end{aligned}
	\end{equation}
	using $G(s) = H(s) - \hat{H}(s)$, due to $\hat{R}_i$ appearing with a negative sign in $G$, and assuming that all of the poles are distinct. We can see that 
	\begin{equation}
		\left.\frac{\diff}{\diff \hat{R}_j} G(-\hat{\lambda}_i)\right|_{\lambda_j} = - \frac{1}{-\hat{\lambda}_i-\lambda_j}
	\end{equation}
	and 
	\begin{equation}
		\left.\frac{\diff}{\diff \hat{\lambda}_j} G(-\hat{\lambda}_i)\right|_{\lambda_j} = \frac{\hat{R}_j}{\left(-\hat{\lambda}_i-\lambda_j\right)^2}
	\end{equation}
	
	Then the conditions are transformed into
	\begin{align*}
		 \underbrace{\sum_{i=1}^{n} R_i \left(- \frac{1}{-{\lambda}_i-\hat{\lambda}_j}\right) - \sum_{i=1}^{d} \hat{R}_i \left(- \frac{1}{-\hat{\lambda}_i-\hat{\lambda}_j}\right)}_{=- G(-\hat{\lambda}_j)} - G(-\hat{\lambda}_j) &= 0
		\quad \mathrm{and}\\
		 \sum_{i=1}^{n} R_i \frac{\hat{R}_j}{\left(-{\lambda}_i-\hat{\lambda}_j\right)^2} - \sum_{i=1}^{d} \hat{R}_i \frac{\hat{R}_j}{\left(-\hat{\lambda}_i-\hat{\lambda}_j\right)^2} &= 0 \quad \quad \mathrm{for} \, j=1,2, \dots d
	\end{align*}
	from the former 
	\begin{equation}
		H(-\hat{\lambda}_i)=\hat{H}(-\hat{\lambda}_i)
	\end{equation}
	from the latter (after dividing by $\hat{R}_j$, and considering that the derivative of $H^\prime(s) = -\sum_{i=1}^{n} \nicefrac{R_i}{\left(s-\lambda_i\right)^2}$)
	\begin{align}
		H'(-\hat{\lambda}_i)&=\hat{H}'(-\hat{\lambda}_i)
	\end{align}
%	\textbf{In the context of Krylov methods for model order reduction, what kind of scheme do these conditions make IRKA?}\\
	These conditions are Hermitian interpolations of $H$, making IRKA the Krylov version of Hermitian interpolation, matching two of the moments at $\sigma_i = -\hat{\lambda}_i$. 
	
	
	%	\textbf{Is it possible to directly make this above condition be satisfied? Why (not)?}\\
	
	Unfortunately, to directly satisfy the first order sufficient conditions for optimality, as we have seen before, we would be required to solve a non-linear system of equations. This would be prohibitively expensive, and possibly ill-conditioned.

	 There is a way to approximate the solution though. \autoref{alg:IRKA} implements this into an algorithm, where we try to converge with the $\sigma_i$ to the optimal $-\hat{\lambda}_i$, for a given size. We project the system into the state and adjoint state space calculated exactly for the current guesses of $\sigma_i$. In this new reduced system we calculate $\hat{\lambda}_i$. It is easy to see, that $\sigma_i = -\hat{\lambda}_i$ would be a fixed point for this procedure (we have seen, that the interpolation would work in interpolation points, which are minus the eigenvalues of the reduced system), and since we showed that state would have optimal cost too, we can hope for convergence (theoretically this is only guaranteed in special cases, i.e. for symmetric positive definite matrices). Convergence is judged based on whether there has been a substantial change in the estimated eigenvalues.
	%\textbf{Use the above reasoning and definitions to explain the basic mechanism behind IRKA, given in algorithm
	%\ref{alg:IRKA} (Observe that this version of IRKA is specifically for descriptor systems)}
	\begin{algorithm2e}[ht]
		\SetKwInOut{Input}{input}
		\SetKwInOut{Output}{output}
		\SetKw{Init}{init}{}{}
		\SetAlgoLined
		\Input{System $(E, A, b, c)$, init. pole estimates $\{\sigma_i\}_{i=1}^{k}$ , tolerance $\epsilon$\\}
		\Output{Approximate model $(\hat{E}, \hat{A}, \hat{b}, \hat{c})$}
		\While{$\text{norm of }\sigma \text{-update}>\text{tol}$}{
			$V=\text{orth}\{\mathbf{x}(\sigma_1),\ldots,\mathbf{x}(\sigma_k)\}$\\
			$W=\text{orth}\{\mathbf{z}(\sigma_1),\ldots,\mathbf{z}(\sigma_k)\}$\\
			$\hat{E}=W^{T}EV, \hat{A}=W^{T}AV, \hat{\mathbf{b}}=W^T\mathbf{b},\hat{\mathbf{c}}=V^T\mathbf{c}$\\
			calculate generalized eigenvalues $\{\hat{\lambda}_1,\ldots,\hat{\lambda}_k\}$ of $(\hat{A},\hat{E})$\\
			set $\sigma_i:=-\hat{\lambda}_i$ for all $i=1,\ldots,k$\\ 
		}
		$V=\text{orth}\{\mathbf{x}(\sigma_1),\ldots,\mathbf{x}(\sigma_k)\}$\\
		$W=\text{orth}\{\mathbf{z}(\sigma_1),\ldots,\mathbf{z}(\sigma_k)\}$\\
		$\hat{E}=W^{T}EV, \hat{A}=W^{T}AV, \hat{\mathbf{b}}=W^T\mathbf{b},\hat{\mathbf{c}}=V^T\mathbf{c}$
		\caption{IRKA}
		\label{alg:IRKA}
	\end{algorithm2e}
	
	With the implementation that has been provided some experiments have been performed on the spiral inductor system in \autoref{fig:sprial_inductor_irka}. Note that due to $\mathcal{H}_2$ optimality, even with $n=1$ the approximation is more accurate in its overall shape. The problem however lies with the speed of the algorithm's convergence, which goes up quite significantly, as the order is increased.
	
	\begin{figure}[h!]
		\centering
		\setlength{\figurewidth}{2\textwidth}
		\tikzsetfigurename{sprial_inductor_irka}
		\input{figs/bode_sprial_inductor_irka.tex}
		\caption{\texttt{IRKA} (as in \autoref{alg:IRKA}) applied to the Spiral Inductor with different orders of approximation.}\label{fig:sprial_inductor_irka}
	\end{figure}
%	\textbf{You have been provided code that implements IRKA for descriptor systems. Adapt this code into a new function that works for quadratic models. What options do you have in case of quadratic models? Which one seems the best? Justify your choice. Apply IRKA to the three models. Compare IRKA to the DPA based approximation schemes.}

\subsection{Greedy Rational Krylov}
%In the slides on model order reduction you can also find the Greedy Rational Krylov method. \textbf{Explain the principles of Greedy rational Krylov. You have been provided a version of greedy rational Krylov for descriptor sytems. Again, as for IRKA, adapt this code into new code for quadratic models. Test this greedy algorithm on the three systems provided and report your results. Compare and contrast as before. Interpret your findings carefully.}
One of the main problems making IRKA and expensive algorithm (too many LU factorizations), is that it tries to converge with all of the eigenvalues all at once. This makes the steps in the algorithm quite expensive, and slow as a result. 

In the Greedy version of IRKA instead of converging with all of the eigenvalues simultaneously, we build the set in a way so that at each iteration step we pick the point with the largest residual norm. Although the set as a whole won't be optimal, and we generally will need more eigenvalues to get the same residue as with the regular IRKA, however the convergence takes less computational effort.

Using the implementation provided for GRKA the estimation has been performed for the spiral inductor in \autoref{fig:sprial_inductor_irka}, where it achieved similar results to the higher order IRKA's, however it has done so with a considerable shorter runtime.
\pdfmargincomment[author=Andras]{I know that the error should be compared here in the H2 sense but I do not have time to implement that.}
\FloatBarrier
\pagebreak
\printbibliography
\end{document}